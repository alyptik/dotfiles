\documentclass[letter]{article} %{{{--

\usepackage[
	    top=.5in,
	    inner=.5in,
	    bottom=.5in,
        left=.3in,
        right=.3in,
	    twoside
	    ]{geometry}

\usepackage{calc}
\usepackage[T1]{fontenc}
\usepackage[utf8]{inputenc}
\usepackage{lmodern}
\usepackage{xcolor,hyperref}
\usepackage{graphicx}
\usepackage{multicol}
\usepackage{xifthen}% provides \isempty test

\usepackage{wasysym} % For phone symbol
\usepackage{eurosym}
\usepackage{url}

\def\bf{\bfseries}
\def\sf{\sffamily}
\def\sl{\slshape}

% Blue colors -------------------------------------------
\definecolor{deep_blue}{rgb}{0,.2,.5}
\definecolor{dark_blue}{rgb}{0,.15,.5}
\definecolor{myblue}{rgb}{.01,0.21,0.71}

% Green colors -------------------------------------------
\definecolor{deep_green}{rgb}{0,.5,.15}
\definecolor{dark_green}{rgb}{0,.45,.1}
\definecolor{mygreen}{rgb}{.01,0.45,0.11}

% Orange colors -------------------------------------------
\definecolor{SkoleBrown}{RGB}{150,100,50}
\definecolor{SkoleYellow}{RGB}{232,183,3}
\definecolor{SkoleOrange}{RGB}{252,135,36}
\colorlet{SkoleLightBrown}{SkoleBrown!50!SkoleYellow}

\colorlet{deep_orange}{SkoleLightBrown!60!black}
\colorlet{dark_orange}{SkoleLightBrown!70!black}
\colorlet{myorange}{SkoleLightBrown!60!black}

% Magenta colors -------------------------------------------
\definecolor{deep_magenta}{rgb}{.35,.1,.35}
\definecolor{dark_magenta}{rgb}{.25,.04,.25}
\definecolor{mymagenta}{rgb}{.3,0.11,0.3}


\colorlet{deep_color}{deep_blue}
\colorlet{dark_color}{dark_blue}
\colorlet{mycolor}{myblue}


\hypersetup{pdftex,  % needed for pdflatex
  breaklinks=true,  % so long urls are correctly broken across lines
  colorlinks=true,
  urlcolor=mycolor,
  linkcolor=dark_color,
  %citecolor=darkgreen,
  }


%% This gives us fun enumeration environments. 
\usepackage{enumitem}

%% More layout: Get rid of indenting throughout entire document
\setlength{\parindent}{0in}

%% Reference the last page in the page number
%
\usepackage{fancyhdr,lastpage}
%\pagestyle{fancy}
\pagestyle{empty}      % Uncomment this to get rid of page numbers
\fancyhf{}\renewcommand{\headrulewidth}{0pt}
\fancyfootoffset{\marginparsep+\marginparwidth}
\lfoot{%\hspace{\footpageshift}%
	\colorlet{mycolor}{black!60}\sffamily%
	\color{black!60}\hfill\, %
		    \qquad\url{http://sangy.xyz}
		    \hfill\hfill
		    Updated \today 
		    \hfill\hfill
                    \arabic{page} of \protect\pageref*{LastPage} %
		    \quad\hfill\,}


\newcommand{\makeheading}[1]%
        {%\hspace*{-\marginparsep minus \marginparwidth}%
         %\begin{minipage}[t]{\textwidth+\marginparwidth+\marginparsep}%
         \begin{minipage}[t]{\textwidth}%
                {\Large #1}\\%[-0.5\baselineskip]%
                 \color{deep_color}{\rule{\columnwidth}{3pt}}%
         \end{minipage}
	 \vskip 1.\baselineskip plus 2\baselineskip minus 1.\baselineskip
	}

\newlength\sidebarwidth
\setlength\sidebarwidth{3.6cm}

\makeatletter
\newcommand{\topic}[3][]%
	 {\pagebreak[3]%
	 \vskip 1.5\baselineskip plus 3\baselineskip minus 0.7\baselineskip
	 \begin{minipage}{\textwidth}
         \phantomsection\addcontentsline{toc}{section}{#2 #1}%
         \nopagebreak\hspace{0in}%
         \nopagebreak\begin{minipage}[t]{\sidebarwidth - .2cm}
         \raggedleft \bf\sf 
	 \color{deep_color}{\Large #2}
	 \end{minipage}%
	 \hfill
	 \begin{minipage}[t]{\linewidth - \sidebarwidth}
	 \nopagebreak{\color{deep_color}%
		    \rule{0pt}{\baselineskip}%
		    \rule{\linewidth}{2.5pt}%
		    \llap{\raisebox{.3\baselineskip}{\sf #1}}%
		    \vspace*{.1\baselineskip}%
		    }%
	 #3%
	 \end{minipage}
	 \end{minipage}}

\newcommand{\smalltopic}[2]%
	 {\pagebreak[2]%
	 \vskip 1\baselineskip plus 2\baselineskip minus 0.3\baselineskip
	 \begin{minipage}{\textwidth}
	 %\hspace{-\marginparsep minus \marginparwidth}%
         \phantomsection\addcontentsline{toc}{subsection}{#1}%
         \nopagebreak\hspace{0in}%
         \nopagebreak\begin{minipage}[t]{\sidebarwidth - .2cm}
         \raggedleft \bf\sf %\vskip -0.5\baselineskip
	 \textcolor{dark_color}{\large #1}%
	 \end{minipage}%
	 \hfill%
	 \ifthenelse{\isempty{#2}}%
	    {%
		    \rule{\linewidth - \sidebarwidth}{.5pt}%
	    }% if #2 is empty
	    {%
	    \begin{minipage}[t]{\linewidth - \sidebarwidth}%
		\nopagebreak{%
		    %\vspace{-.7\baselineskip}%
		    \rule{\linewidth}{.5pt}%
		    \vspace{.1\baselineskip}%
		    }%
		    #2%
		\end{minipage}%
	    }% if #2 is not empty
	 \end{minipage}}

\newcommand{\subtopic}[3][]%
	 {\begin{minipage}{\textwidth}%
	 %\vspace*{.4\baselineskip}%
	 \vskip .4\baselineskip plus 1\baselineskip minus .3\baselineskip
         \nopagebreak\begin{minipage}[t]{\sidebarwidth - .2cm}
	 % Super posh: using semi-bold condensed fonts. Works only with
	 % lmodern
         \raggedleft {\sf\fontseries{sbc}\selectfont #2}
	 {\small\sl\\[-0.2\baselineskip] #1}
	 \end{minipage}%
	 \hfill
	 \begin{minipage}[t]{\linewidth - \sidebarwidth}
	 #3%
	 \end{minipage}%
	 \vspace*{.2\baselineskip plus 1\baselineskip minus
	 .2\baselineskip}%
	 \end{minipage}%
	 \pagebreak[1]%
	 }

\newcommand{\columnlist}[2]%
	 {%
	 \vskip .4\baselineskip plus 1\baselineskip minus .3\baselineskip
	 % Super posh: using semi-bold condensed fonts. Works only with
	 % lmodern
         {\sf\bf ~ ~ #1}\\%
	 \vspace*{-1.2em}%
	 \begin{itemize}[leftmargin=2.8ex, itemsep=0.4ex, parsep=.5ex,
		    itemindent=2ex, labelindent=0ex, labelsep=.5ex]

	 #2%
	 \end{itemize}%
	 \vspace*{.2\baselineskip plus 1\baselineskip minus
	 .2\baselineskip}%
	 \pagebreak[1]%
	 }


\newcommand{\sidenote}[2]
	 {\vspace*{-.2\baselineskip}\begin{minipage}{\textwidth}
         \nopagebreak\hspace{0in}%
         \nopagebreak\begin{minipage}[t]{\sidebarwidth - .2cm}
         \raggedleft {#1}
	 \end{minipage}%
	 \hfill
	 \begin{minipage}[t]{\linewidth - \sidebarwidth}
	 #2%
	 \end{minipage}%
	 \vspace*{.5\baselineskip plus 1\baselineskip minus
	 .2\baselineskip}%
	 \end{minipage}}
\makeatother

% New lists environments 
\newlist{outerlist}{itemize}{1}
\setlist[outerlist]{font=\sffamily\bfseries, label=\textbullet}
\setitemize{topsep=0ex, partopsep=0ex}
\setdescription{font=\normalfont\sffamily\bfseries, itemsep=.5ex,
    parsep=.5ex, leftmargin=3ex}

\newcommand{\blankline}{\quad\pagebreak[2]}

\def\mydot{\textcolor{deep_color}{\rule{1ex}{1ex}}}


% Semi condensed bold
\def\emph#1{{\sf\fontseries{sbc}\selectfont%
	     \textcolor{deep_color!70!black}{#1}}}

%%%%%%%%%%%%%%%%%%%%%%%%%%%%%%%%%%%%%%%%%%%%%%%%%%%%%%%%%%%%%%%%%%%%%--}}}%
\begin{document}
\makeheading{
\begin{minipage}[B]{0.5\textwidth}
    %\vfill
    \vspace*{-.5\baselineskip}%
    \parbox{18cm}{
	\hskip -0.1cm
	{\Huge\bf\sf \color{deep_color} Santiago %
		\color{dark_color} T\huge \hskip -0.05cm ORRES-A\huge \hskip -0.05cm RIAS}
	\\[-.2\baselineskip]
	\bf\sf PhD Candidate in Computer Science
    }
\end{minipage}
\hfill
\begin{minipage}[B]{8cm}
    \raggedleft
    \,\vskip -1em
    \small

    {\texttt {Email: \url{santiago@nyu.edu}}}\\
    {\texttt {GitHub: \href{https://github.com/SantiagoTorres}{github.com/SantiagoTorres}}}\\
    {\texttt {Website: \url{https://badhomb.re/ or https://sangy.xyz}}}

    \vspace*{-.5\baselineskip}%

\end{minipage}
}%
\vspace*{-1em}%

\begin{multicols}{2}
% \sloppy

\textcolor{deep_color}{\bf\sf Research interests}:
Computer Security, Cryptography, Operating Systems, Privacy, Binary and
Malware Analysis\\

\begin{itemize}[leftmargin=2ex, itemsep=0ex]

    \item[\rule{1ex}{1ex}] {\bf I'm an Open Source-oriented developer.} I have
        made contributions to large scale open source projects such as the
        Linux Kernel, Git, Briar, and Mutt/NeoMutt. I'm also a member of the
        Arch Linux CVE Monitoring Team.
        \vspace{1cm}

    \item[\rule{1ex}{1ex}] {\bf Good team work and leadership skills}. I
        currently lead two projects within New York University's Center for
        Cyber Security. I'm the team lead for PolyPasswordHasher and Toto. I'm
        also a main contributor to The Update Framework (TUF) project.


    \item[\rule{1ex}{1ex}] {\bf Dev-ops/Sw engineering research}. My current
        research focuses on elucidating and developing novel ideas to create
        reliable robust software.

\end{itemize}
\end{multicols}
%\fussy

%%%%%%%%%%%%%%%%%%%%%%%%%%%%%%%%%%%%%%%%%%%%%%%%%%%%%%%%%%%%%%%%%%%%%%%%%%%
\topic{E \large\hskip -1ex DUCATION}{~}

    \subtopic[2015--current]{\bf Ph.D}{ 
        Doctor Of Philosophy in Computer Science \\ 
        New York University: Tandon School of Engineering, U.S.,
        {\bf GPA: 3.800}
    }

    \subtopic[2013--2015]{\bf M.S.}{
        Master of Science in Cyber Security\\ 
        New York University: Polytechnic School of Engineering, U.S.,
        {\bf GPA: 4.0}
    }
    \subtopic[2007--2012]{\bf B.S.}{
        Bachelor of Science in Electrical and Telecommunications Engineering,\\
        Universidad Iberoamericana, Mexico, {\bf GPA:} 8.5/10\\
    }
\vspace*{-1\baselineskip}
\smalltopic{Awards}{}

    \subtopic[Research Assistanship]{NYU-TSOE}{
        A research assistanship that covers my full tuition expenses during the
        doctoral program, as well as other living expenses.
    }

    \subtopic[Scholarship]{CONACyT}{
        Mexican Council of Science and Technology Scholarship for outstanding Mexican
        scholars studying abroad. This covers part of my living expenses and the
        total of my tuition.
    }

    \subtopic[Scholarship]{NYU-TSOE}{
        A 8,000 USD per year Merit-Based Scholarship offered by New York University:
        Polytechnic School of Engineering to cover my Master-degree's expenses.  
    }


%   \subtopic[Diploma]{Mexican Governance Secretariat}{
%       Awarded for participating in the Seminar ``Citizens with leadership''.
%   }

    \subtopic[Diploma]{CENEVAL}{
        Diploma of outstanding performance in the Computer Science
        accreditation exam, awarded by the Mexican Evaluation Center.
    }

%%%%%%%%%%%%%%%%%%%%%%%%%%%%%%%%%%%%%%%%%%%%%%%%%%%%%%%%%%%%%%%%%%%%%%%%%%%
% \vspace*{-1.5\baselineskip}
\topic{\mbox{Experience}}{~}

    \subtopic[July 2014--Current.]{Ln(Phi)}{

        \begin{itemize}

            \item Ln(Phi) is a Mexican startup that focuses on research, innovation and 
                implementation of software-related projects (see more at lnphi.com).

            \item I'm the lead security expert. I periodically audit 4 products
                through code reviews, unit and penetration testing.

            \item I'm also the lead of DevOps. With all the team, we elaborate
                new ways to provide continuous integration and continuous
                assurance of the software we develop.

        \end{itemize}

    }


    \hspace{7.5cm}\rule{5cm}{.2pt}

    \subtopic[Apr 2012 -- Aug 2012]{Mexico City Justice Court}{

        \begin{itemize}

            \item Security Audit of the internal networks.

            \item Developer of automatic digitalization systems. I developed an
                automated system for OCR scanning and text embedding for old
                documents.

            \item Via a Liaison, I provided the same service for more than 40
                courts across the city, as well as Mexico's National Forensic
                Service.

            \item I maintained and documented manuals for computer users of different
                levels of expertise.

        \end{itemize}

    }

    \hspace{7.5cm}\rule{5cm}{.2pt}

    \subtopic[Jan 2012 -- May 2012]{Universidad Iberoamericana}{

        \begin{itemize}

            \item High School lecturer: Programming fundamentals.

            \item Taught a classroom of 10 students on how to hack the
                Pokemon red GameBoy game. Through this, they learned
                programming fundamentals such as data structures, constants,
                integrity checks and to never trust your input.

        \end{itemize}

    }


%\vspace*{-\baselineskip}
\smalltopic{Research}{}
    
    \subtopic[July 2015--Current.]{in-toto:\\ Protecting the Software Supply Chain}{

        \begin{itemize}

            \item Version Control System auditing. Analyzed the way multiple
                version control systems store their metadata and how a
                malicious attacker can exploit vulnerabilities in it. This work
                culminated in collaborations with the Git development team.

            \item Design of in-toto. in-toto is a tool to secure the way multiple
                process within the software development lifecycle communicate
                with each other.  Our intention is to provide integrity and
                auditability guarantees from the moment a developer commits a
                line of code to the moment the end user installs a package.

        \end{itemize}

    }

    \hspace{7.5cm}\rule{5cm}{.2pt}

    % FIXME: state what is pph and that I need to work with JC@NYU
    \subtopic[Feb 2014--Current.]{PolyPasswordHasher:\\ A secure Password Storage Mechanism}{
        \begin{itemize}

            % I implemented the whole thing.
            \item PolyPasswordHasher (PPH) uses a Cryptographic Secret Sharing
                Algorithm and Salted Hashing protects password databases so
                that, even if they are stolen, it would take one billion years
                for attackers to crack a single password.
                
            % Why is the django implementation relevant, easy to integrate in
            % real-life/production systems.
            \item I implemented PPH for multiple languages (C, Python) and frameworks 
                (Django) so that organizations -- such as Target and Sony --
                can seamlessly use PPH to protect millions of users.

            \item Leading a research project to measure how the choice of user
                passwords would affect the security provided by PPH in
                real-life scenarios

            \item Lead a team of 5 people with different backgrounds in developing
                new implementations of PPH for new and widespread platforms.


        \end{itemize}
    }

    \hspace{7.5cm}\rule{5cm}{.2pt}

    \subtopic[Sept 2013--current]{The Update Framework:\\A Secure Update System}{
        \begin{itemize}

            \item Provide support, feedback and update TUF's designs.
                Currently, we are working in adapting TUF to car update
                systems.

            \item Designed and built tools for developers to protect software
                packages hosted on the Python Package Index (PyPI).  TUF is
                currently being deployed in bleeding edge technologies such as
                Docker hub and CoreOS.

            \item Designed and conducted experiments to show that Diplomat, a novel 
                security system for protecting community repositories (such as PyPI), 
                protects 99\% of users even if attackers control the repository for a month.


        \end{itemize}
    }

\vspace*{-1.5\baselineskip}
\smalltopic{Publications}{}
    
    % my knowledge about paper taxonomies is very limited, is the correct term
    % here "Article"?

    \subtopic{Conference Paper -- USENIX '16}{
        ``On Omitting Commits and Committing Omissions: Preventing Git metadata
        tampering that (re)introduces vulnerabilities''. {\bf S. Torres-Arias},
        A. Kumar Ammula, R. Curtmola, J. Cappos. 
    }

    \subtopic{Conference Paper -- NSDI '16}{
        ``Diplomat: Using Delegations to Protect Community Repositories''. T. K. Kuppusamy,
        V. Diaz, {\bf S. Torres-Arias}, J. Cappos. 
    }

    \subtopic{Conference Paper -- Under peer review }{
    ``\{PASS\}: Gaining the Higher Ground Against Password Crackers'' 
      J. Cappos, {\bf S. Torres-Arias}. (ACSAC 2016)
    }

    \subtopic{Article}{
        ``PolyPasswordHasher: Improving Password Storage Security'', Login; Magazine,
        Security issue. {\tt December 2014}.
    }
%   \vspace{-1\baselineskip}
%   \smalltopic{Industry}{\\[-.3\baselineskip]}
%   \vspace{-1\baselineskip}

%       \subtopic[Mexico, Jan--May 2013]{Federal District's Superior Justice Court}{
%           \begin{itemize}

%               \item Integrated open source tools for process automation and
%                   performed network and host security assessments to servers that
%                   hosted sensitive data.

%           \end{itemize}
%       }


\vspace*{-1\baselineskip}
\smalltopic{Talks \& Posters}{}

    \subtopic{Talk}{
        ``PolyPasswordHasher: no password left behind'', Army Cyber Institute, ``Cyber Talks'',
        {\tt September 2016}.
    }

    \subtopic{Poster}{
        ``PolyPasswordHasher: no password left behind'', New England Networking and Systems Day,
        {\tt October 2014}.


    }
 

\vspace*{-1\baselineskip}
\smalltopic{Volunteering}{}

    \subtopic[Nov 2015- current]{The Briar Project}{
        \begin{itemize}

            \item The Briar project is a distributed messaging application for
                first responders and activists. I'm an occasional commiter to
                the project.

            \item I revamped the ``introduction client'' which is used to add
                contacts in the network by using a trusted middleman.

            \item Refactored of Tor-related UI elements. I updated modules for
                connectivity via tor.

            \item I Refactored user interfaces in general. Translated current
                programmatic user interfaces to XML, so the project's designers
                could easily edit them.

        \end{itemize}
    }


    \hspace{7.5cm}\rule{5cm}{.2pt}

    \subtopic[May 2016 - current]{NeoMutt}{
        \begin{itemize}

            \item Mutt is a command line Mail User Agent (much like Thunderbird
                or outlook).  I'm an official developer of it's successor,
                NeoMutt.

            \item I maintain the new-mail feature. Which is the notification
                handler for when a new mail arrives.

        \end{itemize}
    }


    \hspace{7.5cm}\rule{5cm}{.2pt}

    \subtopic[July 2015 - current]{Arch Linux}{
        \begin{itemize}

            \item Arch Linux is one of the most popular and robust Linux
                Distributions.  I'm member of the Arch Linux CVE Monitoring
                Team.

            \item I constantly notify developers of new vulnerabilities, as
                well as interact with vendors for appropriate disclosure and
                patching of vulnerabilities. I'm also part of the team that
                writes and elaborates the Arch Security Advisories (ASA), to
                notify the community and IT experts of security updates for
                Arch Linux's packages. You can follow us on
                \url{https://twitter.com/arch\_security}

            \item I'm also part of the testing team. I verify that packages
                are correct before they are released to the general public.

        \end{itemize}
    }


    \hspace{7.5cm}\rule{5cm}{.2pt}

    \subtopic[Feb 2009 - May 2014]{Universidad Iberoamericana}{
        \begin{itemize}

            \item Co-founded the Software-development group. We Focus on
                teaching introductory programming and computer science
                to undergraduate students.

            \item Creator of
                \href{https://github.com/GIDAIbero/MATSOL/}{MATSOL}, an
                \href{https://itunes.apple.com/us/app/matsol/id389393518?mt=8}{iOS}
                and
                \href{https://play.google.com/store/apps/details?id=gidaibero.android.matsol}{Android}
                toolkit for engineering students. On its good days, MATSOL used
                to be the top-25 most downloaded educational application on
                spanish-speaking countries.

        \end{itemize}
    }

%%%%%%%%%%%%%%%%%%%%%%%%%%%%%%%%%%%%%%%%%%%%%%%%%%%%%%%%%%%%%%%%%%%%%%%%%%%
\vspace*{-2\baselineskip}
\topic{Technical Skills}{}
\begin{multicols}{2}
    \paragraph{Languages}{
        \begin{itemize}
            \item Spanish (Native proficiency), English (107 TOEFL IBT), French (B1 DELFT).
        \end{itemize}
    }
    \vspace*{-1\baselineskip} 
    \paragraph{Programming Languages}{
        \begin{itemize}
            % these are your stronger languages, don't say you are knowledgeable
            % say you are proficient. 
            \item C, Python, Java, shellscript (bash/zsh variants), ASM (x86,
                PIC, MSP430), MATLAB/Octave, Objective-C, PHP, SQL and C++.

            \item {\bf Frameworks/API's/etc:}
                Django, OpenGL, Repy, pydasm, iOS, Android, Travis
                (CI). 
                
            \item Systemd Unit writing, system tracing, automated debugging,
                and application profiling.

            \item {\bf Operating Systems:}
                Linux (Debian, RedHat, Arch, LFS variants), Windows, MacOS, Minix.

            \item {\bf Favorite tools}: Vim, Git, tmux, neomutt, gdb, automake, radare2.

        \end{itemize}
    }

\end{multicols}
\end{document}
